%-------------------------
% Resume in Latex
% Author : Jake Gutierrez
% Based off of: https://github.com/sb2nov/resume
% License : MIT
%------------------------

\documentclass[letterpaper,11pt]{article}

\usepackage{latexsym}
\usepackage[empty]{fullpage}
\usepackage{titlesec}
\usepackage{marvosym}
\usepackage[usenames,dvipsnames]{color}
\usepackage{verbatim}
\usepackage{enumitem}
\usepackage[hidelinks]{hyperref}
\usepackage{fancyhdr}
\usepackage[english]{babel}
\usepackage{tabularx}
\usepackage{fontawesome5}
\usepackage{multicol}
\setlength{\multicolsep}{-3.0pt}
\setlength{\columnsep}{-1pt}
\input{glyphtounicode}


%----------FONT OPTIONS----------
% sans-serif
% \usepackage[sfdefault]{FiraSans}
% \usepackage[sfdefault]{roboto}
% \usepackage[sfdefault]{noto-sans}
% \usepackage[default]{sourcesanspro}

% serif
% \usepackage{CormorantGaramond}
% \usepackage{charter}


\pagestyle{fancy}
\fancyhf{} % clear all header and footer fields
\fancyfoot{}
\renewcommand{\headrulewidth}{0pt}
\renewcommand{\footrulewidth}{0pt}

% Adjust margins
\addtolength{\oddsidemargin}{-0.6in}
\addtolength{\evensidemargin}{-0.5in}
\addtolength{\textwidth}{1.19in}
\addtolength{\topmargin}{-.8in}
\addtolength{\textheight}{1.4in}

\urlstyle{same}

\raggedbottom
\raggedright
\setlength{\tabcolsep}{0in}

% Sections formatting
\titleformat{\section}{
  \vspace{-4pt}\scshape\raggedright\large\bfseries
}{}{0em}{}[\color{black}\titlerule \vspace{-5pt}]

% Ensure that generate pdf is machine readable/ATS parsable
\pdfgentounicode=1

%-------------------------
% Custom commands
\newcommand{\resumeItem}[1]{
  \item\small{
    {#1 \vspace{-2pt}}
  }
}

\newcommand{\classesList}[4]{
    \item\small{
        {#1 #2 #3 #4 \vspace{-2pt}}
  }
}

\newcommand{\resumeSubheading}[4]{
  \vspace{-2pt}\item
    \begin{tabular*}{1.0\textwidth}[t]{l@{\extracolsep{\fill}}r}
      \textbf{#1} & \textbf{\small #2} \\
      \textit{\small#3} & \textit{\small #4} \\
    \end{tabular*}\vspace{-7pt}
}

\newcommand{\resumeSubSubheading}[2]{
    \item
    \begin{tabular*}{0.97\textwidth}{l@{\extracolsep{\fill}}r}
      \textit{\small#1} & \textit{\small #2} \\
    \end{tabular*}\vspace{-7pt}
}

\newcommand{\resumeProjectHeading}[2]{
    \item
    \begin{tabular*}{1.001\textwidth}{l@{\extracolsep{\fill}}r}
      \small#1 & \textbf{\small #2}\\
    \end{tabular*}\vspace{-7pt}
}

\newcommand{\resumeSubItem}[1]{\resumeItem{#1}\vspace{-4pt}}

\renewcommand\labelitemi{$\vcenter{\hbox{\tiny$\bullet$}}$}
\renewcommand\labelitemii{$\vcenter{\hbox{\tiny$\bullet$}}$}

\newcommand{\resumeSubHeadingListStart}{\begin{itemize}[leftmargin=0.0in, label={}]}
\newcommand{\resumeSubHeadingListEnd}{\end{itemize}}
\newcommand{\resumeItemListStart}{\begin{itemize}}
\newcommand{\resumeItemListEnd}{\end{itemize}\vspace{-5pt}}

%-------------------------------------------
%%%%%%  RESUME STARTS HERE  %%%%%%%%%%%%%%%%%%%%%%%%%%%%


\begin{document}

%----------HEADING----------
% \begin{tabular*}{\textwidth}{l@{\extracolsep{\fill}}r}
%   \textbf{\href{http://sourabhbajaj.com/}{\Large Sourabh Bajaj}} & Email : \href{mailto:sourabh@sourabhbajaj.com}{sourabh@sourabhbajaj.com}\\
%   \href{http://sourabhbajaj.com/}{http://www.sourabhbajaj.com} & Mobile : +1-123-456-7890 \\
% \end{tabular*}

\begin{center}
    {\Huge \scshape Kunal Kumar Sahoo} \\ \vspace{1pt}
    Gandhinagar, Gujarat \\ \vspace{1pt}
    \small \raisebox{-0.1\height}\faPhone\ +91 74330 64468 ~ \href{mailto:kunal.sahoo2003@gmail.com}{\raisebox{-0.2\height}\faEnvelope\  \underline{kunal.sahoo2003@gmail.com}} ~ 
    \href{https://linkedin.com/in//}{\raisebox{-0.2\height}\faLinkedin\ \underline{/in/kunal-kumar-sahoo}}  ~
    \href{https://github.com/}{\raisebox{-0.2\height}\faGithub\ \underline{Kunal-Kumar-Sahoo}}
    \vspace{-8pt}
\end{center}


%-----------EDUCATION-----------
\section{Education}
  \resumeSubHeadingListStart
    \resumeSubheading
      {Pandit Deendayal Energy University}{2021 -- 2025}
      {B.Tech in Computer Engineering (CGPA: \textbf{9.96})}{Gandhinagar, Gujarat}
    \resumeSubheading  
      {Kendriya Vidyalaya No.1, Gandhinagar}{2014 -- 2021}
      {Class XII (Percentage: \textbf{94.8\%})}{Gandhinagar, Gujarat}
  \resumeSubHeadingListEnd

%-----------PROGRAMMING SKILLS-----------
\section{Technical Skills}
 \begin{itemize}[leftmargin=0.15in, label={}]
    \small{\item{
     \textbf{Languages}{: Python (NumPy, Pandas, Matplotlib, Scikit learn, Tensorflow, Keras, MediaPipe, OpenCV), Java, C, C++, Dart, MySQL} \\
     \textbf{Developer Tools}{: VS Code, Jupyter Notebooks, Google Colab, Android Studio, ViM, SolidWorks, TinkerCAD} \\
     \textbf{Technologies/Frameworks}{: Linux, GitHub, Flutter, Flask, Arduino, Raspberry PI} \\
    }}
 \end{itemize}
 \vspace{-16pt}


%-----------EXPERIENCE-----------
\section{Experience}
  \resumeSubHeadingListStart

    \resumeSubheading
      {SLoP 2.0, DA-IICT}{Oct 2022 -- Present}
      {Open source contributor}{Remote}
      \resumeItemListStart
        \resumeItem{Worked on the project \textbf{Finding data from Permanent Shadowed Regions}}
        \resumeItem{Tech-stack: Python (NumPy, Pandas, Matplotlib), OpenCV .}
        \resumeItem{Utilizing the concepts of \textbf{Image Processing}, \textbf{Machine Learning} and \textbf{Deep Learning} to find data in the permanent shadowed regions by processing OHRC images}
        \resumeItem{Developing a Python script to generate \textbf{.tif images} from the data provided by \textbf{Indian Space Science Data Centre} }
      \resumeItemListEnd

    \resumeSubheading
      {Encode, PDEU}{August 2022 -- Present}
      {App Developer}{Gandhinagar, Gujarat}
      \resumeItemListStart
        \resumeItem{Develop \textbf{Cross-platform mobile applications} for other departments and clubs in university. }
        \resumeItem {Use \textbf{Flutter} which is a framework on Dart language developed by Google.}
        \resumeItem{Currently developing a \textbf{full-stack} mobile application for the quiz club in a \textbf{team} of 3 people.}
        \resumeItem{Collaborated with team members using version control systems such as \textbf{Git} and services like Trello to organize modifications and assign tasks.}
        \resumeItem{Utilized \textbf{Android Studio} as a development environment in order to develop and test the working of the applications.}
    \resumeItemListEnd
    
    \resumeSubheading
      {Cretus, PDEU}{August 2021 -- Present}
      {Technical member}{Gandhinagar, Gujarat}
      \resumeItemListStart
        \resumeItem{Develop \textbf{robots} and \textbf{IoT gadgets} for various kinds of competitions and exhibitions. }
        \resumeItem {Use technologies like \textbf{Arduino}, \textbf{Raspberry Pi}, \textbf{Linux}, and softwares like \textbf{Arduino IDE}, \textbf{SolidWorks} and \textbf{TinkerCAD} for developing projects. }
        \resumeItem{Currently building a \textbf{miniature Mars rover} for the technical fest of the university.}
        \resumeItem{Have conducted workshops on \textbf{Getting started with Arduino Development}}
        \resumeItem{Previously built competitive robots like \textbf{Line-following robot}, \textbf{Maze solver}, etc.}
    \resumeItemListEnd
    
  \resumeSubHeadingListEnd
\vspace{-16pt}

%-----------PROJECTS-----------
\section{Projects}
    \vspace{-5pt}
    \resumeSubHeadingListStart
      \resumeProjectHeading
          {\textbf{SitRight} $|$ \emph{Python, MediaPipe, OpenCV, SQLite, Raspberry Pi}}{July 2022}
          \resumeItemListStart
            \resumeItem{Developed a Python-based \textbf{IoT device} that connects to monitors and alerts user when bad sitting posture is maintained too long at \textbf{Pythakon'22 hackathon}  held at CHARUSAT.}
            \resumeItem{Implemented the solution on a Raspberry Pi which can connect to user's workstation (Desktop/Laptop) wirelessly.}
            \resumeItem{Used \textbf{OpenCV} for extracting image frames from video feed and \textbf{MediaPipe} to detect and skeletonize human body from the image and determine key points}
            \resumeItem{Developed a simple \textbf{mathematical model} to determine angle of inclination between the back and neck and used \textbf{Machine learning algorithms} to classify the posture as 'Good' or 'Bad' and send alert messages and report to user's workstation}
          \resumeItemListEnd
          \vspace{-13pt}
      \resumeProjectHeading
          {\textbf{Face-Mask-Detection} $|$ \emph{Python, Tensorflow, OpenCV}}{January 2022}
          \resumeItemListStart
            \resumeItem{Developed a Python software that can classify in \textbf{real-time} whether a person is wearing a mask or not}
            \resumeItem{Implemented \textbf{MobileNetV2} architecture which contained \textbf{convolution layer} with 32 filters, followed by 19 \textbf{residual bottleneck filters}}
            \resumeItem{Used \textbf{Kaggle} dataset for training the model }
          \resumeItemListEnd 
          \vspace{-13pt}
      
    \resumeSubHeadingListEnd
\vspace{-15pt}



\end{document}
